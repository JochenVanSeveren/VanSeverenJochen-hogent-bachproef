%---------- Inleiding ---------------------------------------------------------

% TODO: remove this

% Dag Manu en Guillaume!
 
% Het is een maandje geleden dat ik Manu eens navroeg welke bachelor proeven jullie mij konden aanbieden en daar vermeldde ik dat ik graag nog even op zoek wou gaan naar een onderwerp dat mij persoonlijk echt aanspreekt en een bijdrage zou brengen aan mijn omgeving. 
 
% Context: 
 
% Sinds dan, werd ik passief overvallen door het kwijt geraken van mijn kat en de hele zoektocht door facebookgroepen op zoek naar hem en ik wou dit tracking probleem oplossen, maar daar bestaat al een website voor, dus ging ik verder op zoek : )
 
% Een weekje geleden kwam Thomas Bonte, de founder van MuseScore (open source muzieknotatiesoftware) spreken in mijn keuzevak ondernemen. Omdat ik zelf jarenlang van deze software heb kunnen genieten, dacht ik mijn kans te grijpen en te vragen of ik hem kon helpen met een bachelor proef, maar helaas, hij heeft het bedrijf al verkocht.

% Er kwam ter sprake dat ik mijn skills wou inzetten voor sustainable bedrijven en niet rap voor bijvoorbeeld de verfindustrie waar ik lang studentejobs heb gedaan in de fabriek en in de laboratoria. Toen hij verfindustrie en mijn ervaring ermee hoorde, begon hij direct over Clean Water Global.
% Ik quoteer eens de founder van CWG zijn laatste LinkedIn post om een schets te geven van wat ze doen:

% 🔍 Hoewel het verboden is,duidelijk gemeld op veiligheidsfiches, worden er dagelijks enorme hoeveelheden vervuild verfspoelwater geloosd in de riool. Met bestanddelen als microplastics,acticide,biocides,titaniumdioxiden, pigmenten en zware metalen is het meer dan duidelijk dat verfspoelwater ver weg moet blijven van zeeën, oceanen en drinkwater.

% Tijdens ons pilotjaar 2022/2023 verzamelden we met 80 schilders,40.000liter verfspoelwater. Daaruit filterde we 6,2ton verfresidu. 
% Kortom 1 schilder 2000liter water en 355kg verfresidu. Dat staat gelijk aan bijna een volledige pallet met verf die hierdoor niet in onze riolering terecht kwam,per persoon.

% Het residu word voorbereid als circulaire grondstof en ontwikkeld voor re-integratie in coatings,decoratieve technieken en plamuren.

% Met Clean Water Global ontwikkelden we een systeem om vervuiling door verfspoelwater tegen te gaan: een oplossing waarmee we samen met alle schilders, verfwinkels en verfproducenten bijdragen aan een circulaire,duurzame en propere toekomst! 🌏
 
% Ik heb Timmy Bours direct gecontacteerd met een nog langere mail dan deze met de vraag of zijn bedrijf voor technische problemen staat waarvoor IT een oplossing zou kunnen bieden.
% Bijna direct heb ik kunnen online meeten met hem.
 
% Idee:
 
% Van wat ik kon verstaan uit de eerste meeting, is dit hun werkwijze:
% Er staan meerdere depositpunten opgesteld waar schilders kunnen betalen om hun verfafval water af te zetten en te laten filteren. Hiervoor betalen ze een fee aan de partner waarbij het depositpunt opgesteld staat en deze partner betaald dan Clean Water Global om deze te komen leeghalen (deze factuur wordt manueel opgemaakt in Odoo en de volheid van de tanks worden bijgehouden met een monitoringssysteem dat een sms verstuurd, maar dit blijkt niet optimaal te zijn).
% De schilders houden deze facturen bij om aan te tonen dat ze in orde zijn met de regelgeving.
 
% Ik heb dit voorgelegd aan de bachelorproef begeleider bij Hogent en ze zei me dat een complex probleem oplossen voor een bedrijf, een ideale bachelorproef is mits genoeg analyse en onderzoek van wat er mogelijk is.
 
% Concreet kwamen we op het volgende:
% Een app waar schilders hun deposits kunnen bijhouden door middel van certificaten per bezoek en een globaal overzicht van hoeveel ze al gedeponeerd hebben. Met dat schilders vaak voor een bedrijf werken, kunnen we onmiddellijk ook voor het volledige bedrijf de voetafdruk tonen. 
% Het zou interessant zijn om te onderzoeken in hoeverre we kunnen helpen om te bewijzen aan de overheid of ze in lijn zijn met de huidige regelgeving.
% Hiervoor is er ook een systeem nodig om de partner bedrijven het aantal liters van de deposittransactie te laten inputten.
 
% Dit volstond volgens de docent als bachelorproef al, maar ik heb nog wat extra ideëen die ik nog zou kunnen toevoegen wanneer ik mijn bachelorproefvoorstel uitschrijf:
 
% Als extensie hiervan bood ik hem de solutie aan om de invoices naar de schilders niet meer met een middleman (de partner waar het depositpunt staat) te laten verlopen, maar door rechtstreeks in de app de betalingsoptie aan te bieden en dan in 1 keer de middleman uit te betalen. 
% Odoo zijn xmlRPC laag ken ik al dus ik zou erin slagen om customers en invoices aan te maken in Odoo en mogelijks zelfs Stripe (nog onderzoek voor nodig) verbinden om direct betaling uit te voeren.
% Het zou een meerwaarde zijn om meer mensen de regelgeving te laten volgen moesten bedrijven voor al hun schilders in dienst, een balans kunnen toevoegen in de app als depositbudget.  
 
% Timmy Bours sprak ook over de monitors in de depositpunten te verbeteren, maar ik heb nog geen hardware kennis kunnen op doen.
 
% Voorstel:
 
% Mijn vraag is nu of dit binnen we-are past en of jullie dit zouden zien zitten om mij in te begeleiden?
% Ik gokte van wel, want ik heb jullie gekozen als stageplek omdat ik het gevoel heb dat jullie ook aan missies willen bijdragen en dat er heel wat software architectuur kennis in het team zit wat ideaal is voor deze bp en om bij te leren in een stage.
% Het idee klinkt misschien nog wat breed, maar na jullie feedback, kan ik het nog wat meer afbakenen of dingen weglaten.
 
% Het was een beetje een lange mail, maar het is ook al even een zoektocht geweest naar een goed idee voor mij dus ik hoop dat jullie mijn enthousiasme delen!
 
% Indien er interesse is, hebben jullie tijd voor een meeting deze week?
% Erna meet ik nog eens met Timmy en schrijf ik het voorstel concreet uit (deadline 15 december).
 
% Met vriendelijke groeten,
% Jochen Van Severen
 
% jochenvanseveren.com
 

\section{Introductie}%
\label{sec:introductie}

Dit onderzoek vertrekt vanuit een concrete casus, 
maar behandelt de bredere onderzoeksvraag: 
Hoe kan de integratie tussen moderne cross-platform applicaties en ERP-systeem Odoo worden gerealiseerd in het kader van verfafvalwaterdeposits door schilders?

% TODO: BUG wrap line above
\bigskip
Voor mijn bachelorproef ging ik op zoek naar een bedrijf wiens missie ik deel 
en waar technisch onderzoek nodig is om hun werking te verbeteren.
In mijn zoektocht kwam ik terecht bij Clean Water Global, 
een bedrijf dat zich inzet voor een circulaire economie door verfafvalwater te filteren en te hergebruiken.
Het probleem dat het bedrijf tracht op te lossen is het illegaal lozen van verfafvalwater door schilders.
% TODO: BRON AAN TIMMY VRAGEN 
Cijfers uit het pilotjaar van Clean Water Global tonen aan dat men voor 80 schilders, 40.000 liter verfspoelwater kon verzamelen en hieruit 6,2 ton verfresidu filteren.
Het voorgaande gebruik om dit afvalwater illegaal te dumpen is in huidige jaren sterk gereguleerd. 
De ondernememing Clean Water Global biedt een manier aan om de schilders in regel te stellen met de huidige wetgeving.
Ze doen dit door een netwerk van partners op te stellen die hun depositpunten plaatsgeven waar schilders dan hun verfafvalwater kunnen afzetten.
Dit proces verloopt als volgt: de schilder betaalt een fee aan de partner waarbij het depositpunt opgesteld staat en deze partner betaald dan Clean Water Global om deze te komen leeghalen.
Bij dit factureren komt er veel administratie te kijken bij de partner telkens en 
is er naast papieren factuurbonnen ook geen duidelijk overzicht van hoeveel deze al gedeponeerd heeft 
voor de schilder in questie en bij extensie het bedrijf waar deze werkt.

\bigskip

Clean Water Gebruikt huidig het open source ERP-systeem Odoo om hun facturatie te beheren en deze bachelorproef zal onderzoeken hoe er een integratie kan opgezet worden 
tussen Odoo en een cross-platform framework om een mobiele applicatie te ontwikkelen die de schilders faciliteert in het betalen van hun deposits en het bijhouden van hun certificaten.
Deze integratie zal het mogelijk maken om de facturatie rechtstreeks met de schilder te laten verlopen en de partner uit te betalen in 1 keer.
Hiervoor is er een authenticatie systeem om de schilders te laten inloggen, een systeem nodig om het correcte aantal liters van de deposittransactie te laten inputten, 
een manier om betaling uit te voeren en een manier om de certificaten te laten bijhouden zodat de schilders kunnen aantonen dat ze in orde zijn met de regelgeving.

\bigskip 

Het onderzoek zal focussen op wat er functioneel mogelijk is en hier een proof of concept van ontwikkelen.
% TODO: insert independant study how many companies use ODOO OR market share

%---------- Stand van zaken ---------------------------------------------------

\section{State-of-the-art}%
\label{sec:state-of-the-art}



%---------- Methodologie ------------------------------------------------------
\section{Methodologie}%
\label{sec:methodologie}

% IDEA SECTION: 
% literatuurstudie naar bestaande cross platform frameworks, en dan een vergelijkende studie uitvoeren om te bepalen welk framework het beste is voor de casus
% literatuurstudie naar bestaande odoo modules
% interview bij mensen bij Odoo / Edwin CTO Lab box en zijn team
% contact OCA

% Vergelijking Fastapi in Odoo / Fastapi server / Fastapi serverless

% Requirements analyse bij Clean Water Global adhv interview eerste dondedag bp

%---------- Verwachte resultaten ----------------------------------------------
\section{Verwacht resultaat, conclusie}%
\label{sec:verwachte_resultaten}

