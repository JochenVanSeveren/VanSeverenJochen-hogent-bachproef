%---------- Inleiding ---------------------------------------------------------
\section{Introductie}%
\label{sec:introductie}

Dit onderzoek vertrekt vanuit een concrete casus, 
maar behandelt de bredere onderzoeksvraag: 
Hoe kan de integratie tussen moderne cross-platform applicaties en ERP-systeem Odoo worden gerealiseerd in het kader van verfafvalwaterdeposits door schilders?

% TODO: BUG wrap line above
\bigskip
Voor mijn bachelorproef ging ik op zoek naar een bedrijf wiens missie ik deel 
en waar technisch onderzoek nodig is om hun werking te verbeteren.
In mijn zoektocht kwam ik terecht bij Clean Water Global, 
een bedrijf dat zich inzet voor een circulaire economie door verfafvalwater te filteren en te hergebruiken.
Het probleem dat het bedrijf tracht op te lossen is het illegaal lozen van verfafvalwater door schilders.
% TODO: BRON AAN TIMMY VRAGEN 
Cijfers uit het pilotjaar van Clean Water Global tonen aan dat men voor 80 schilders, 40.000 liter verfspoelwater kon verzamelen en hieruit 6,2 ton verfresidu filteren.
Het voorgaande gebruik om dit afvalwater illegaal te dumpen is in huidige jaren sterk gereguleerd. 
De ondernememing Clean Water Global biedt een manier aan om de schilders in regel te stellen met de huidige wetgeving.
Ze doen dit door een netwerk van partners op te stellen die hun depositpunten plaatsgeven waar schilders dan hun verfafvalwater kunnen afzetten.
Dit proces verloopt als volgt: de schilder betaalt een fee aan de partner waarbij het depositpunt opgesteld staat en deze partner betaald dan Clean Water Global om deze te komen leeghalen.
Bij dit factureren komt er veel administratie te kijken bij de partner telkens en 
is er naast papieren factuurbonnen ook geen duidelijk overzicht van hoeveel deze al gedeponeerd heeft 
voor de schilder in questie en bij extensie het bedrijf waar deze werkt.

\bigskip

Clean Water Gebruikt huidig het open source ERP-systeem Odoo om hun facturatie te beheren en deze bachelorproef zal onderzoeken hoe er een integratie kan opgezet worden 
tussen Odoo en een cross-platform framework om een mobiele applicatie te ontwikkelen die de schilders faciliteert in het betalen van hun deposits en het bijhouden van hun certificaten.
Deze integratie zal het mogelijk maken om de facturatie rechtstreeks met de schilder te laten verlopen en de partner uit te betalen in 1 keer.
Hiervoor is er een authenticatie systeem om de schilders te laten inloggen, een systeem nodig om het correcte aantal liters van de deposittransactie te laten inputten, 
een manier om betaling uit te voeren en een manier om de certificaten te laten bijhouden zodat de schilders kunnen aantonen dat ze in orde zijn met de regelgeving.

\bigskip 

Het onderzoek zal focussen op wat er functioneel mogelijk is en hier een proof of concept van ontwikkelen.
% TODO: insert independant study how many companies use ODOO OR market share

%---------- Stand van zaken ---------------------------------------------------

\section{State-of-the-art}%
\label{sec:state-of-the-art}

Odoo is relatief nieuw, maar wel een van de meest gebruikte ERP-systemen al.
Dit heeft als gevolg dat er weinig tot geen onderzoek is gedaan naar de integratie van Odoo met cross-platform frameworks.
Voorbeelden zijn schaars te vinden, maar aangezien Odoo een open source project is, is de opportuniteit er wel om zelf een integratie te ontwikkelen.
Onderzoek heeft een kleine kans om uit te wijzen dat dit niet mogelijk is door bijvoorbeeld te veel security risico's, maar dit is onwaarschijnlijk.

\subsection{Odoo}%
\label{subsec:odoo}
Odoo, voorheen OpenERP, is een open-source Enterprise Resource Planning (ERP) systeem dat geëvolueerd is tot een uitgebreide suite van business management tools.
Volgens de studie van \textcite{GaneshEtAl2016} vindt men een volgende uitgebreide beschrijving van Odoo.
Het werd geïntroduceerd in een belangrijke update met de release van Odoo/OpenERP 8.0.
Dit systeem omvat niet alleen traditionele ERP functionaliteiten, maar integreert ook bijkomende features zoals een Content Management System (CMS), e-commerce mogelijkheden en business intelligence tools.
\bigskip
Odoo's architectuur maakt gebruik van verschillende open-source technologieën, wat flexibiliteit en schaalbaarheid garandeert.
Het gebruikt Python als zijn primaire programmeertaal, gekend voor zijn eenvoud en compatibiliteit over platformen heen.
De database server van het systeem is PostgreSQL, die object-relationele mapping ondersteunt en zeer schaalbaar is in termen van data opslag en gebruikerscapaciteit.
Odoo maakt gebruik van het MVC framework, wat de scheiding van data (model) en user interface (view) faciliteert, met een controller als tussenschakel.
\bigskip
Een belangrijk aspect van Odoo is zijn kosteneffectiviteit en aanpasbaarheid, wat het toegankelijk maakt voor bedrijven van verschillende groottes, van kleine tot grote industrieën.
De open-source aard van Odoo laat absolute eigendom van het systeem toe door organisaties, wat volledige controle over de broncode geeft en de afhankelijkheid van leveranciers vermindert.
Deze feature is bijzonder voordelig op vlak van aanpassingen, onderhoud en het upgraden van het systeem.
De kwaliteitsborging in Odoo is hoog door de actieve betrokkenheid en bijdragen van een community van onafhankelijke ontwikkelaars.
\bigskip
Odoo maakt een deel van zijn functionaliteiten beschikbaar via een API zoals gedocumenteerd op hun website\cite{odooExternalApi}.
De communicate verloopt via xml remote procedure calls (xmlRPC) en is beschikbaar in verschillende programmeertalen\cite{xmlrpcWhatXMLRPC}

\subsection{Cross-Platform Frameworks}%
\label{subsec:cross-platform}
Deze studie is niet gericht op het ontwikkelen van een mobiele applicatie, maar op het onderzoeken van de integratie tussen Odoo en een cross-platform framework.
Toch is het belangrijk om een goed beeld te hebben van wat er mogelijk is met cross-platform frameworks.
Hier volgt een korte beschrijving van de twee van de meest gebruikte cross-platform frameworks waartussen gekozen zal worden.

\paragraph{Flutter:}
Flutter is een open-source UI toolkit ontwikkeld door Google.
Het is een cross-platform framework dat gebruikt kan worden om native applicaties te ontwikkelen voor Android, iOS, Windows, Mac, Linux en web.
Flutter maakt gebruik van de Dart programmeertaal, die ook ontwikkeld is door Google.
Dart is een object-georiënteerde programmeertaal die gebruik maakt van een garbage collector en een ahead-of-time (AOT) compiler.
De AOT compiler zet Dart code om naar native machine code voor de doelplatformen.

\paragraph{React Native:}
React Native is een open-source framework ontwikkeld door Facebook.
Het is een cross-platform framework dat gebruikt kan worden om native applicaties te ontwikkelen voor Android, iOS, Windows en web.
React Native maakt gebruik van de JavaScript programmeertaal.
JavaScript is een object-georiënteerde programmeertaal die gebruik maakt van een garbage collector en een just-in-time (JIT) compiler.
De JIT compiler zet JavaScript code om naar native machine code voor de doelplatformen.

\bigskip

De keuze tussen deze twee frameworks zal gemaakt worden op basis van de resultaten van de literatuurstudie en de requirements analyse.


\subsection{Api framework}%
\label{subsec:api_framework}

Hoe de api best gebouwd zal worden en waar deze best gehost wordt, zal onderzoek uitwijzen, maar hier komt er toch al een kort stukje om enkele opties uit te leggen.

\paragraph*{Oca (Odoo Community Association): Fastapi module}
De Odoo Community Association is een non-profit organisatie die de ontwikkeling van Odoo modules ondersteunt.
Ze hebben een module ontwikkeld die het mogelijk maakt om fastapi te draaien in Odoo~\autocite{ocaRestFramework2023}
Dit heeft als voordeel dat de api in Odoo zelf kan draaien en dus geen extra infrastructuur nodig heeft, 
maar Maxime Van Reyd (Projectmanager bij Odoo) heeft aangegeven dat ze deze optie niet verkiezen.

\paragraph*{Dedicated server}
Maxime Van Reyd en andere contactpersonen bij Odoo hebben liever een server die buiten Odoo draait om de maintainability en upgradability van Odoo te garanderen.
Dit heeft als voordeel dat de api niet in Odoo zelf draait en dus geen impact heeft op de performantie van Odoo, maar dit heeft als nadeel dat er een extra server nodig is.

\paragraph*{Serverless functions}
Serverless functions zoals de Azure functions van Microsoft~\textcite{microsoftAzureFunctions} zijn een optie om de integratie zo kost efficiënt mogelijk te maken, 
maar hebben ook hun nadelen die verder in de paper besproken zullen worden.

\bigskip

Over de effectieve intergratie tussen Odoo en de cross-platform framework is er weinig tot geen onderzoek te vinden.
Dit is een van de redenen waarom dit onderzoek interessant is.

%---------- Methodologie ------------------------------------------------------
\section{Methodologie}%
\label{sec:methodologie}

% Voorbeeld RM
%     A[Literatuur] --> B[Requirements];
%     A --> C[Long list];
%     B --> D[Short list];
%     C --> D;
%     D --> E[Proof-of-Concept];
%     E --> F[Conclusies];
%     F --> G[Scriptie];


% IDEA SECTION: 
% literatuurstudie naar bestaande cross platform frameworks, en dan een vergelijkende studie uitvoeren om te bepalen welk framework het beste is voor de casus
% literatuurstudie naar bestaande odoo modules
% interview bij mensen bij Odoo / Edwin CTO Lab box en zijn team
% contact OCA

% Vergelijking Fastapi in Odoo / Fastapi server / Fastapi serverless

% Requirements analyse bij Clean Water Global adhv interview eerste dondedag bp

De methodologie van van dit onderzoek zal als volgt verlopen:

\begin{enumerate}
    \item \textbf{Literatuurstudie:}
    \begin{itemize}
        \item Analyse van bestaande Odoo-modules en API-frameworks, inclusief de Fastapi module van OCA.
        \item Interviews met professionals bij Odoo, Lab Box en OCA voor inzichten in beste praktijken en aanbevelingen.
        \item Vergelijkend onderzoek tussen cross-platform frameworks, voornamelijk Flutter en React Native.
    \end{itemize}
    
    \item \textbf{Requirements Analyse:}
    \begin{itemize}
        \item Analyse van de bedrijfsprocessen en requirements bij Clean Water Global en het afbakenen van de scope van de integratie. 
        \item Opstellen van een longlist en shortlist van potentiële oplossingen op basis van literatuurstudie en requirements analyse.
    \end{itemize}

    \item \textbf{Selectie van Technologieën:}
    \begin{itemize}
        \item Opstellen van een longlist en shortlist van potentiële oplossingen op basis van literatuurstudie en requirements analyse.
        \item Beoordeling van de haalbaarheid en efficiëntie van verschillende benaderingen: Fastapi binnen Odoo, dedicated server of serverless functies.
    \end{itemize}

    \item \textbf{Ontwikkeling van een Proof-of-Concept:}
    \begin{itemize}
        \item Implementatie van een basisversie van de applicatie in het gekozen framework, geïntegreerd met Odoo.
        \item Testen op functionaliteit, prestaties en gebruiksgemak.
    \end{itemize}

    \item \textbf{Analyse en Conclusie:}
    \begin{itemize}
        \item Evaluatie van de proof-of-concept en formuleren van aanbevelingen voor de integratie.
    \end{itemize}
\end{enumerate}

%---------- Verwachte resultaten ----------------------------------------------
\section{Verwacht resultaat, conclusie}%
\label{sec:verwachte_resultaten}

Dit onderzoek is gericht op het vaststellen van een efficiënte en effectieve manier om Odoo te integreren met een cross-platform mobiele applicatie. De verwachte resultaten omvatten:

\begin{itemize}
    \item Een gedetailleerd inzicht in de sterke en zwakke punten van Flutter en React Native in relatie tot hun integratie met Odoo.
    \item Aanbevelingen voor de meest geschikte technologieën en frameworks, afgestemd op de requirements van Clean Water Global.
    \item Een functionerend proof-of-concept dat de integratie in de praktijk demonstreert.
    \item Praktische richtlijnen voor bedrijven die vergelijkbare integraties overwegen.
\end{itemize}

De conclusie zal richtlijnen bieden over de meest efficiënte en effectieve aanpak voor de integratie, rekening houdend met zowel technische haalbaarheid als bedrijfsbehoeften. Dit zal waardevolle inzichten bieden voor Clean Water Global en kan dienen als referentiekader voor andere bedrijven.
